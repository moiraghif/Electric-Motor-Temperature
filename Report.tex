%%%%%%%%%%%%%%%%%%%%%%%%%%%%%%%%%%%%%%%%% 
% University Assignment Title Page 
% LaTeX Template
% Version 1.0 (27/12/12)
%
% This template has been downloaded from:
% http://www.LaTeXTemplates.com
%
% Original author:
% WikiBooks (http://en.wikibooks.org/wiki/LaTeX/Title_Creation)
%
% License:
% CC BY-NC-SA 3.0 (http://creativecommons.org/licenses/by-nc-sa/3.0/)
% 
% Instructions for using this template:
% This title page is capable of being compiled as is. This is not useful for 
% including it in another document. To do this, you have two options: 
%
% 1) Copy/paste everything between \begin{document} and \end{document} 
% starting at \begin{titlepage} and paste this into another LaTeX file where you 
% want your title page.
% OR
% 2) Remove everything outside the \begin{titlepage} and \end{titlepage} and 
% move this file to the same directory as the LaTeX file you wish to add it to. 
% Then add \input{./title_page_1.tex} to your LaTeX file where you want your
% title page.
%
%%%%%%%%%%%%%%%%%%%%%%%%%%%%%%%%%%%%%%%%%
%\title{Title page with logo}
%----------------------------------------------------------------------------------------
%	PACKAGES AND OTHER DOCUMENT CONFIGURATIONS
%----------------------------------------------------------------------------------------

\documentclass[12pt]{article}
\usepackage[english]{babel}
\usepackage[utf8x]{inputenc}
\usepackage{amsmath}
\usepackage{graphicx}
\usepackage{hyperref}
\usepackage{enumitem}
\usepackage{minted}
\usepackage[colorinlistoftodos]{todonotes}

\begin{document}

\begin{titlepage}

\newcommand{\HRule}{\rule{\linewidth}{0.5mm}} % Defines a new command for the horizontal lines, change thickness here

\center % Center everything on the page
 
%----------------------------------------------------------------------------------------
%	HEADING SECTIONS
%----------------------------------------------------------------------------------------

\textsc{\LARGE Università degli studi di Milano-Bicocca}\\[1cm] % Name of your university/college
\textsc{\Large Advanced Machine Learning }\\[0.3cm] % Major heading such as course name
\textsc{\large Final Project}\\[0.1cm] % Minor heading such as course title

%----------------------------------------------------------------------------------------
%	TITLE SECTION
%----------------------------------------------------------------------------------------

\HRule \\[0.4cm]
{ \huge \bfseries Electical Motor Temperature}\\[0.2cm] % Title of your document
\HRule \\[1.5cm]
 
%----------------------------------------------------------------------------------------
%	AUTHOR SECTION
%----------------------------------------------------------------------------------------

\large
\emph{Authors:}\\
Federico Moiraghi - 799735 - f.moiraghimotta@campus.unimib.it \\
Pranav Kasela - 846965 - p.kasela@campus.unimib.it \\
Roberto Berlucchi - xxxxxx - r.xx@@campus.unimib.it \\
[0.5cm] % Your name

% If you don't want a supervisor, uncomment the two lines below and remove the section above
%\Large \emph{Author:}\\
%John \textsc{Smith}\\[3cm] % Your name

%----------------------------------------------------------------------------------------
%	DATE SECTION
%----------------------------------------------------------------------------------------

{\large 2019-2020}\\[1cm] % Date, change the \today to a set date if you want to be precise

%----------------------------------------------------------------------------------------
%	LOGO SECTION
%----------------------------------------------------------------------------------------

\includegraphics{imgs/logo.png}\\[1cm] % Include a department/university logo - this will require the graphicx package
 
%----------------------------------------------------------------------------------------

\vfill % Fill the rest of the page with whitespace

\end{titlepage}


\begin{abstract}
  The present Report is a summary of methodologies used to predict the temperature of various part of a prototype electrical motor given some tests on a bench.
  The resulting model have to yield acceptable predictions and to be light enough to be used by the car itself during its daily use: autos can start cooling components before the temperature grows critically (first task) or can estimate temperature without a specific sensor (second task), due to its cost and weakness, knowing only basic environmental information.
\end{abstract}

\section{Introduction}
The data set comprises several sensor data collected from a permanent magnet synchronous motor (PMSM) deployed on a test bench.
The PMSM represents a German OEM's prototype model. 
Test bench measurements were collected by the LEA department at Paderborn University.

The recordings are sampled at a frequency of $2Hz$ and is divided in various profiles and has a total of $998070$ observations. 
Each profile indicates a different session and each session can have different length varying from one to six hours.\\
The input variables are:
\begin{itemize}[topsep=0ex, noitemsep]
    \item \textbf{Ambient temperature} as measured by a thermal sensor located close to the stator;
    \item \textbf{Coolant temperature}, the motor is water cooled and the measurement is taken at outflow;
    \item The current and voltage are transformed through a $dq0$ transformation in a d-q coordinate system, it basically converts a three phase balanced reference system (in an AC system) into 2 coordinates, denoted by d and q, via a rotating reference frame with angle $\theta$.
    The currents are denoted by \textbf{i\_d} and \textbf{i\_q} and the voltages are denoted by \textbf{u\_d} and \textbf{u\_q};
    \item \textbf{Motor speed}.
\end{itemize}
The target variables are:
\begin{itemize}[topsep=0ex, noitemsep]
    \item \textbf{pm}: Permanent Magnet surface temperature representing the rotor temperature, measured with an infrared thermography unit.
    \item \textbf{stator\_yoke}, \textbf{stator\_tooth}, \textbf{stator\_winding} temperature measured with a thermal sensor of the corresponding components.
\end{itemize}
In some of the variables, gaussian noise is introduced to simulate real world driving cycles.
Being sensors data, missing values are replaced by the provider with the previous one, causing some flat areas when sensors fall offline for a long period. \\
The main objective is to create a lightweight model to predict the \verb|pm| and \verb|stator| variables minimizing the MSE (because the model needs to be deployed with best cost-precision ratio); a secondary objective is to predict more accurately higher temperature than the lower temperature using a modified loss.

\section{Datasets}
The data set can be found on Kaggle\footnote{\href{https://www.kaggle.com/wkirgsn/electric-motor-temperature}{https://www.kaggle.com/wkirgsn/electric-motor-temperature}}.
From the data set the \verb|torque| feature is immediately excluded, as it is considered unreliable from the data set provider itself.
\begin{figure}[!h]
    \centering
    \includegraphics[width=\linewidth, height=10cm]{imgs/dist_plot.png}
    \caption{Distribution of the variables grouped by the division}
    \label{fig:dist_plot}
\end{figure}\\
The data set is divided into train, validation and test set: validation data consists of \verb|profile_id|s $20, 31, 46, 54, 62, 70, 79, 72$, the test set profiles are $35$ and $42$ and the training set consists of all the other profiles.
Their relative distributions are plotted in the Figure \ref{fig:dist_plot}.\\
In the Figure \ref{fig:corrplot} the correlation between the variables is shown.
The target variables are highly correlated among themselves, in particular the \verb|stator| variables.
\begin{figure}[!h]
    \centering
    \includegraphics[width=\linewidth, height=7cm]{imgs/corrplot.png}
    \caption{Correlation Plot of the considered variables}
    \label{fig:corrplot}
\end{figure}\\
Data was already standardized by the provider (for some anonymization issues), but variables do not have a normal distribution thus a further normalization between 0 and 1 (calculated only on the training set) is applied.

For each profile the data set is modified: to each row the lagged features is concatenated, the number of lags to keep can be changed, after this procedure the rows of all the profiles are united and shuffled through a custom data loader.

\section{The Methodological Approach}
Different \textit{Deep Learning} architectures are tried in order to compare them and choose the most suitable model to the problem: each model built using \mintinline{python}|pytorch| and optimized with an Auto-ML algorithm using \mintinline{python}|sherpa| optimization library, in particular the GP surrogate model is used with the EI acquisition function to have some exploration during the optimization.

All models are implemented as Sequence to Value (Seq2Val) so that the training process can be paralleled and shuffled, with improvements in both speed and quality.
In addition, a Sequence to Value model can be easily rewritten as Sequence to Sequence, providing real-time information to the driver.


\paragraph{RNN}
In order to accomplish both tasks with a lightweight model easily usable by a car, a simple Recurrent Neural Network is implemented: this model is provided with one hidden layer that depends on both current data and previous observations, and uses it to estimate the target variables.
After the hidden layer, data flows through two independent feed-forward neural networks (with two layers both) to estimate values for \verb|pm| and \verb|stator|s variables: this approach is justified by the higher correlation between \verb|stator| temperatures and lower correlation with \verb|pm|.

Which layer (the previous input, the estimated output or the embedding) the RNN yields to the next step and how to use this information (where it is linked) are parameters settable in the model construction and optimized through Auto-ML; in addition, \textit{learning rate}, number of neurons per hidden layer, and length of the input sequence are optimized in the same way: to do so, the \mintinline{python}|Custom_RNN.forward()| method is implemented recursively to make it as adaptable as possible.

\paragraph{LSTM and GRU}
Two more type of recursive neural architecture are tested, the first one uses a LSTM layer while the other uses a GRU layer instead.
In this case two independent models are created: one to predict only the \verb|pm| variable and one to predict the 3 \verb|stator|s variables.

In this case the window is fixed to 60 lags and it does not need to be optimized, since the LSTM and GRU should automatically understand how many lags are important.
The model is pretty simple (as required by the task), it has one recursive layer (either LSTM or GRU) followed by two fully connected layer, the second fully connected layer can be omitted if the number of neurons are 0, lastly there is the output layer which has one neuron while predicting \verb|pm| and three neurons while predicting \verb|stator|s.

The hyperparameter optimizer need to optimize the number of hidden units, the number of neurons for each fully connected layer, the learning rate and the batch size of the data.

\paragraph{CNN}





% This is the central and most important section of the report. Its objective must be to show, with linearity and clarity, the steps that have led to the definition of a decision model. The description of the working hypotheses, confirmed or denied, can be found in this section together with the description of the subsequent refining processes of the models. Comparisons between different models (e.g. heuristics vs. optimal models) in terms of quality of solutions, their explainability and execution times are welcome. 

%inifinitive generator!?
% You should also mention any unforeseen problems you encountered when implementing the system and how and to what extent you overcame them. Common problems are:  difficulties involving existing software.


 \section{Results and Evaluation}
 
\subsection{First Task}
As shown in figure \ref{fig:automl_mse}, after the optimization process LSTM and GRU yield similar results (having a similar structure), while CNN provides results comparable with the previous architecture.
RNN however yields poor previsions, due to its simplicity, compared to other architectures.
\begin{figure}[!h]
    \centering
    \includegraphics[width=\linewidth]{imgs/comparison_MSE.png}
    \caption{Results of the optimization process (RNN is excluded due to its poor performance)}
    \label{fig:automl_mse}
\end{figure}\\

However, comparing results using only the loss value is not satisfying for the task: the model have to be light enough to be easily used by a car: so architectures are plotted considering both number of parameters and performance.
% TODO: performance ~ numero di parametri
  
% The Results section is dedicated to presenting the actual results (i.e. measured and calculated quantities), not to discussing their meaning or interpretation. The results should be summarized using appropriate Tables and Figures (graphs or schematics). Every Figure and Table should have a legend that describes concisely what is contained or shown. Figure legends go below the figure, table legends above the table. Throughout the report, but especially in this section, pay attention to reporting numbers with an appropriate number of significant figures. 

\section{Discussion}
% The discussion section aims at interpreting the results in light of the project's objectives. The most important goal of this section is to interpret the results so that the reader is informed of the insight or answers that the results provide. This section should also present an evaluation of the particular approach taken by the group. For example: Based on the results, how could the experimental procedure be improved? What additional, future work may be warranted? What recommendations can be drawn?


\section{Conclusions}
% Conclusions should summarize the central points made in the Discussion section, reinforcing for the reader the value and implications of the work. If the results were not definitive, specific future work that may be needed can be (briefly) described. The conclusions should never contain ``surprises''. Therefore, any conclusions should be based on observations and data already discussed. It is considered extremely bad form to introduce new data in the conclusions.

\section*{References}

% The references section should contain complete citations following standard form.  The references should be numbered and listed in the order they were cited in the body of the report. In the text of the report, a particular reference can be cited by using a numerical number in brackets as \cite{Lee2015} that corresponds to its number in the reference list. \LaTeX provides several styles to format the references

\bibliographystyle{IEEEtran}
% \bibliography{references.bib}

\end{document}